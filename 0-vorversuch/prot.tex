\documentclass{scrreprt}

\KOMAoptions{fontsize=11pt, paper=a4} % Schriftgröße und Papierformat setzen.
\KOMAoptions{DIV=16} % Parameter mit dem man den Seitenrand ändern kann.


\usepackage{mathtools}
\usepackage{amssymb}
\usepackage{hyperref}
\usepackage[all]{hypcap}

% common macros so i don't have to import packages or type too much
\def\diff#1#2{\frac{\text{d}#1}{\text{d}#2}}
\def\diffp#1#2{\frac{\partial#1}{\partial#2}}
\def\imply{\Longrightarrow}
\def\d#1{\text{d}#1\text{ }}
\def\equiv{\Longleftrightarrow}

\def\jafp#1#2{
    \begin{center}
%    \includegraphics[width=0.8\textwidth]{#1}
	\includegraphics[height=0.4\textheight]{#1}
    \captionof{figure}{#2}
    \label{fig:#1}
    \end{center}
}

\newcommand\disclaimer{
	\chapter*{Vorbemerkungen}
	Dieses Protokoll wurde gemeinsam von erstellt und (außer uns sind Fehler bei der Versionierung unterlaufen) zwei mal gleich abgegeben. Quellcode (auch \LaTeX) verfügbar auf \url{https://github.com/byteOfWisdom/the-ep-cant-hurt-you}. Schaltbilder ohne explizite Quelle sind mit Tikz erzeugt, Diagramme ohne explizite Quelle mit Python (Oder gnuplot. Oder Julia.).
}

\addtokomafont{chapterprefix}{\raggedleft}
\addtokomafont{chapter}{\Large}
\addtokomafont{section}{\large}
\addtokomafont{subsection}{\large}
\addtokomafont{subsubsection}{\large}

\usepackage{graphicx}
\graphicspath{{./plots/}}
\author{Leonie Dessau \& Carla Vermöhlen}
\setlength\parindent{0pt}


\title{Vorversuch}

\begin{document}
\maketitle
\tableofcontents
\disclaimer

\chapter{Voraufgaben}
\section{A}
Mit $U(t) = U_0 \sin(\omega t)$ ist trivial $U_{ss} = 2 U_0$ und $U_s = U_0$.
Weiterhin ist:

\begin{align*}
	U_{eff}^2 &= \langle U_0^2 \sin(\omega t)^2 \rangle = \lim_{x\to\infty} \frac{U_0^2}{T} \int_0^T \d t \sin(\omega t)^2\\
	&= \lim_{x\to\infty} \frac{U_0^2 T}{2 T} - \frac{\sin(2 \omega T)}{2 T} = \frac{U_0^2}{2}\\
	\imply U_{eff} &= \frac{U_0}{\sqrt{2}}
\end{align*}

\section{B}
Sei ein symmetrisches rechtecksignal gegeben durch $U(t) = U_0 (2 \Theta(\sin(\omega t)) - 1)$, dann ist $U_{eff}$:
\begin{align*}
	U_{eff}^2 = \frac{1}{T} \int_0^T \d t U_0^2 (2 \Theta(\sin(\omega t)) - 1)^2
\end{align*}
Nun wird $T = \frac{1}{\omega}$ gewählt (also über genau eine Schwingung integriert).

\begin{align*}
	U_{eff}^2 &=  \frac{1}{\omega}\int_0^{\frac{1}{2\omega}} \d t U_0^2 + \int_{\frac{1}{2\omega}}^{\frac{1}{\omega}} \d t U_0^2 = U_0^2 \\ \imply& U_{eff} = U_0
\end{align*}


\section{C}
Zunächst ist festzuhalten, dass der Strom entlang aller Bauelemente gleich, also $I_i = I_n$, ist und das Ohmsche Gestz hier als $U = R I$ geschrieben werden kann.
\begin{align*}
	&U_n = U_0 \frac{R_n}{R_n + R_i}\\
	\equiv& U_0 = \frac{R_n + R_i}{R_n} U_n = (R_n + R_i) I_n
\end{align*}
Da $U_0$ konstant ist, kann $U_0$ von zwei verschiedenen Messungen gleichgesetzt werden:

\begin{align*}
	&U_0 = (R_1 + R_i) I_1 = (R_2 + R_i) I_2\\
	\equiv& R_1 I_1 + R_i I_1 = U_1 + R_i I_1 = R_2 I_2 + R_i I_2 = U_2 + R_i I_2\\
	\equiv& U_1 - U_2 = R_i I_2 - R_i I_1 = R_i (I_2 - I_1)\\
	\imply& R_i = \frac{U_1 - U_2}{I_2 - I_1}
\end{align*}

Es ist für den Funktionsgenerator bekannt, dass für $I_1 = 0$ $U_{1, ss} = 20V$ ist und bei $I_2 = 50\Omega$ $U_{2, ss} = 10$ ist. Setzt man dies in die obige Formel ein, erhält man $R_i = \frac{20V - 10V}{50\Omega} = 0.2\Omega$.


\section{D}



\end{document}
