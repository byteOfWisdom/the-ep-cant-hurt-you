\documentclass{scrreprt}

\KOMAoptions{fontsize=11pt, paper=a4} % Schriftgröße und Papierformat setzen.
\KOMAoptions{DIV=12} % Parameter mit dem man den Seitenrand ändern kann.


\usepackage{mathtools}
\usepackage{amssymb}
\usepackage{hyperref}
\usepackage[all]{hypcap}
\usepackage{subcaption}
\usepackage{float}
\usepackage[german]{babel}
\usepackage[separate-uncertainty = true,multi-part-units=brackets, number-unit-product = \text{ }, mode=math, exponent-product = \cdot, output-product = \cdot]{siunitx}
\usepackage{ circuitikz }

% common macros so i don't have to import packages or type too much
\def\diff#1#2{\frac{\text{d}#1}{\text{d}#2}}
\def\diffp#1#2{\frac{\partial#1}{\partial#2}}
\def\imply{\Longrightarrow}
\def\d#1{\text{d}#1\text{ }}
\def\equiv{\Longleftrightarrow}

\def\jafp#1#2{
    \begin{center}
    \includegraphics[width=0.8\textwidth]{#1}
	%\includegraphics[height=0.4\textheight]{#1}
    \captionof{figure}{#2}
    \label{fig:#1}
    \end{center}
}

\def\jafps#1#2#3{
    \begin{center}
    \includegraphics[width=#3\textwidth]{#1}
	%\includegraphics[height=0.4\textheight]{#1}
    \captionof{figure}{#2}
    \label{fig:#1}
    \end{center}
}


\def\jafpp#1#2#3#4{
    \begin{figure}[H]
    \centering
    \begin{subfigure}{0.45\textwidth}
    \centering
    \includegraphics[width = \textwidth]{#1}
    \caption{}
    \label{fig:#1}
    \end{subfigure}
    \begin{subfigure}{0.45\textwidth}
    \centering
    \includegraphics[width = \textwidth]{#3}
    \caption{}
    \label{fig:#3}
    \end{subfigure}
    \caption{a) #2 und b) #4}
    \end{figure}
    }


\newcommand\disclaimer{
	\chapter*{Vorbemerkungen}
	Dieses Protokoll wurde gemeinsam von Carla Vermöhlen und Leonie Dessau erstellt und (außer uns sind Fehler bei der Versionierung unterlaufen) zwei mal gleich abgegeben. Quellcode (auch \LaTeX) verfügbar auf \url{https://github.com/byteOfWisdom/the-ep-cant-hurt-you}. Schaltbilder ohne explizite Quelle sind mit Tikz erzeugt, Diagramme ohne explizite Quelle mit Python (Oder gnuplot. Oder Julia.). Die Signaldiagramme wurden mit den csv Dateien aus dem Oszilloskop geplottet.
}

\addtokomafont{chapterprefix}{\raggedleft}
\addtokomafont{chapter}{\Large}
\addtokomafont{section}{\large}
\addtokomafont{subsection}{\large}
\addtokomafont{subsubsection}{\large}


\makeatletter
\renewcommand\chapter{\thispagestyle{plain}%
\global\@topnum\z@
\@afterindentfalse
\secdef\@chapter\@schapter}
\makeatother

\usepackage{graphicx}
\graphicspath{{./plots/}}
\author{Leonie Dessau \& Carla Vermöhlen}
\setlength\parindent{0pt}
