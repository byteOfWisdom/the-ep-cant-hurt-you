\documentclass{scrreprt}

\KOMAoptions{fontsize=11pt, paper=a4} % Schriftgröße und Papierformat setzen.
\KOMAoptions{DIV=16} % Parameter mit dem man den Seitenrand ändern kann.


\usepackage{mathtools}
\usepackage{amssymb}
\usepackage{hyperref}
\usepackage[all]{hypcap}

% common macros so i don't have to import packages or type too much
\def\diff#1#2{\frac{\text{d}#1}{\text{d}#2}}
\def\diffp#1#2{\frac{\partial#1}{\partial#2}}
\def\imply{\Longrightarrow}
\def\d#1{\text{d}#1\text{ }}
\def\equiv{\Longleftrightarrow}

\def\jafp#1#2{
    \begin{center}
%    \includegraphics[width=0.8\textwidth]{#1}
	\includegraphics[height=0.4\textheight]{#1}
    \captionof{figure}{#2}
    \label{fig:#1}
    \end{center}
}


\usepackage{graphicx}
\graphicspath{{./plots/}}
\author{Leonie Dessau \& Carla Vermöhlen}
\setlength\parindent{0pt}


\title{Versuch 1: Ausbreitung von Signalen auf Leitern}

\begin{document}
\maketitle
\tableofcontents
\disclaimer

\chapter{Einleitung}

\chapter{Theorie}
\textbf{Leitungseigenschaften:} Koaxialkabel, wie in diesem Versuch, bestehen aus einem Innen- und einem Außenleiter (z.B. Volldraht und Kupfer), die durch ein Dielektrikum voneinander abgeschirmt sind. Sie besitzen also eine Induktivität und eine Kapazität, für die gilt
\begin{align}
    C = \epsilon_r \epsilon_0 l \frac{2\pi}{\ln{\frac{R_a}{R_i}}} \\
    L = \mu_r \mu_0 l \frac{\ln{\frac{R_a}{R_i}}}{2\pi}
\end{align}

Wobei l die Länge des Leiters, $\text R_a$ der Radius des Außenmantels, $\text R_i$ der Radius des inneren Leiters sind. $ \epsilon_r \text{ und } \mu_r$ die elektrischen und magnetischen Permeabilitäten sind Materialkonstanten.
$\epsilon_0 \text{, } \text{, } \mu_0 $ sind die Vakuumpermeabilitäten. \\
Es gibt außerdem noch zwei weitere Kenngrößen für eine den Widerstand R und den Verlustwert G.
Die Größen sind proportional zur Länge des Leiters $l$ und werden deshalb pro Längeneinheit angegeben, also als $ \text {R'} = \frac {\text R}{l} \text{; } \text {C'} = \frac {\text C}{l} \text{; } \text {L'} = \frac {\text L}{l} \text{ und }\text {G'} = \frac {\text G}{l}$.
Diese vier Leitungskonstanten führen zu weiteren Leitungsgrößen: dem Wellenwiderstand, der Verzögerungszeit und der Dämpfung. \\ \\

\textbf{Wellenausbreitung auf homogenen Leitern:} Um die Vorgänge in den verwendeten Koaxialkabeln zu beschreiben, wird das Kabel als verlustlose Leitung betrachtet, welche aus aneinandergereihten LC-Gliedern besteht, betrachtet. Zur Beschreibung von Strom und Spannung nutzt man die Impedanz und Admittanz aus folgendem Ersatzschaltbild: \\

\jafp{Ersatzschaltbild01.png}{Schaltbild aus dem Skript}

Die Längsimpedanz ist folglich eine Serienschaltung aus Spule und Verlustwiderstand und die Queradmittanz eine Reihenschaltung aus Kondensator und Verlustleitwert.

\begin{align}
    Z = i\omega L + R \\
    Y = i\omega C + G
\end{align}

Auch diese Größen sind Proportional zur Länge und werden dementsprechend pro Längeneinheit angegeben.

\begin{align}
    Z'= \frac{Z}{l} = \frac{\Delta Z}{\Delta l} \\
    Y'= \frac{Y}{l} = \frac{\Delta Y}{\Delta l}
\end{align}

Des Weiteren ist über Strom und Spannung aus dem Ersatzschaltbild bekannt

\begin{align}
    U(x) &= I(x) \cdot \Delta Z + U(x+\Delta x) \\
    I(x) &= I(x+\Delta x) + U(x + \Delta x) \cdot \Delta Y
\end{align}

Man erhält folgendes Differentialgleichungssystem, wenn man $\Delta U \text{ und } \Delta I$ durch $\Delta x$ teilt und dann $\Delta x \rightarrow 0$ laufen lässt.

\begin{align}
    \frac{dU}{dx} = -I \cdot Z' \\
    \frac{dI}{dx} = -U \cdot Y'
\end{align}

Durch weiteres Ableiten erhält man die Lösung des DGL Systems. Die Spannung wird durch die Superrposition einer hin- und rücklaufenden Spannungsamplitude und der Strom durch die Differenz dieser bestimmt.

\begin{align}
    U(x,t) = U_h(x,t) + U_r(x,t) = (U_{h0} e^{- \Upsilon x} + U_{r0} e^{\Upsilon x}) e^{i\omega t} \\
    I(x,t) = (U_{h0} e^{- \Upsilon x} - U_{r0} e^{\Upsilon x}) \cdot \sqrt{\frac{G'+i\omega C'}{R'+i\omega L'}} e^{i\omega t} = I_h(x,t) + I_r(x,t)
\end{align}

Wobei für die Dämpfung $\Upsilon $ gilt:
\begin{align}
    \Upsilon ^2 = Z' \cdot Y' = (R' + i\omega L') \cdot (G' + i\omega C') = \alpha + i\beta
\end{align}
mit $\alpha = \Re  (\Upsilon)$ der Dämpfungskonstante. Im verlustfreien Fall ($R'= G'=  0 $) wird $\Upsilon = i\omega \sqrt{L'C'} = i\beta$. \\

\textbf{Wellenwiderstand und Phasengeschwindigkeit:} Die Wellenlänge, die Phasengeschwindigkeit und die Gruppengeschwindigkeit lauten:

\begin{align}
    \lambda &= \frac{2\pi}{\beta} = \frac{2\pi}{\omega \sqrt{L'C'}} \\
    v_{ph} &= \nu \cdot \lambda = \frac{\omega}{\beta} = \frac{1}{\sqrt{L'C'}} = c_0 \cdot \frac{1}{\sqrt{\epsilon _r \mu _r}}\\
    v_{gr} &= \frac{d\omega}{d\beta} = \frac{1}{d\beta / d\omega} = \frac{1}{\sqrt{L'C'}} = v_{ph}
\end{align}

Der Wellenwiderstand ist folgendermaßen definiert

\begin{align}
    Z = \frac{U_h(x)}{I_h(x)} = \frac{U_{h0}}{I_{h0}} = \sqrt{\frac{R'+i\omega L'}{G'+i\omega C'}} = \sqrt{\frac{L'}{C'} \cdot \sqrt{\frac{1-i\frac{R'}{\omega L'}}{1- i\frac{G'}{\omega C'}}}} = \frac{U_r(x)}{-I_r(x)}
\end{align}

Für den verlustfreien Fall wird
\begin{align}
    Z = \sqrt{\frac{\mu_r \mu_0}{\epsilon_r \epsilon_0}} \frac{\ln(R_a / R_i)}{2 \pi} = \sqrt{\frac{L'}{C'}} &= Z_{frei} \frac{\ln(R_a / R_i)}{2 \pi}
\end{align}

Die Verzögerungszeit des Kabels ist antiproportional zur Phasengeschwindigkeit $v_{ph}$. In Aufgabe A und B werden ihre Eigenschaften näher behandelt. \\

\textbf{Leitungsabschluss und Anpassung:} Im Kabel gibt es wie oben beschrieben sowohl eine einlaufende als auch eine rücklaufende Welle. Das liegt daran, dass die Energie der einlaufenden Welle nicht komplett im Abschlusswiderstand $ R_A $ verbraucht wird. Wenn der Abschlusswiderstand genau gleich dem Wellenwiderstand ist, gibt es demnach keine rücklaufende Welle.\\
Der Abschlusswiderstand ergibt sich aus dem Ohm'schen Gesetz zu

\begin{align}
    R_A = \frac{U_h(l)+U_r(l)}{I_h(l)+I_r(l)} = Z \cdot \frac{U_{hl}+U_{rl}}{U_{hl}-U_{rl}} = Z \cdot \frac{1+r}{1-r}
\end{align}

Hierbei ist $r$ der Reflexionsfaktor

\begin{align}
    r = \frac{U_{rl}}{U_{hl}} = \frac{1-\frac{Z}{R_A}}{1+\frac{Z}{R_A}} = \frac{R_A-Z}{R_A+Z}
\end{align}

Außerdem relevante Größen sind das Stehwellenverhältnis $s$ und der Anpassungsfaktor $m$

\begin{align}
    s = \frac{1+|r|}{1-|r|} \\
    m = \frac 1s
\end{align}

Es ergeben sich drei verschiedene Fälle für die Auswirkung des Abschlusswiderstandes auf die Signale: \\

\textbf{Angepasster Abschluss:} $R_A = Z \text{; } r = 0 \text{; } s = 1 \text{ und } m= 1 $ \\
In diesem Fall gibt es keine Reflexion, also auch keine rücklaufende Welle. Die gesamte einlaufende Energie wird an den Verbraucher abgegeben.\\

\textbf{Offene Leitung:} $ R_A = \infty \text{; } r = +1 \text{; } s = \infty \text{ und } m = 0 $ \\
Der Strom am Ende des Leiters ist $ I_{rl} + I_{hl = 0}$, da nach außen kein Strom fließt. Es gilt also $ I_{rl} = -I_{hl}$ und damit $U_{hl} = U_{rl}$. Die hinlaufende Welle ist genauso groß, wie die rücklaufende Welle. Außerdem ist die hinlaufende Welle so groß, wie die hinlaufende Welle im Fall des angepassten Anschlusses.\\

\textbf{Kurzschluss:} $ R_A = 0 \text{; } r = -1 \text{; } s = \infty \text{ und } m = 0 $ \\
Am Ende des Leiters wird die Spannung 0, da hier kurzgeschlossen ist. $ U_{rl} + U_{hl} = 0 $ also $ U_{hl} = - U_{rl}$. Die rücklaufende Welle ist also genau entgegengesetzt gleich groß, wie die hinlaufende Welle. Der Strom ist doppelt so groß, wie bei angepasstem Anschluss. $I_{hl} + I_{rl} = 2 I_{hl}$

\chapter{Voraufgaben}

\section{A}
Größere Verzögerungszeiten lassen sich erreichen indem man ein längeres Kabel verwendet oder, indem man die Phasengeschwindigkeit reduziert. Betrachtet man
\begin{align}
    v_{ph} = \frac{1}{\sqrt{L' C'}} = c_0 \frac{1}{\sqrt{\epsilon_r \mu_r}}\label{vph}
\end{align}\sgl{vph}{1.13}
so kann man diese reduzieren indem $L' C'$ vergrößert wird. Dies lässt sich erreichen indem entweder die Kapazität erhöht oder die Induktivität. Dies lässt sich erreichen, indem man Materialien mit größeren elektrischen- oder magnetischen Permeabilitäten nutzt.

\section{B}
Der Wellenwiderstand ist gegeben als
\begin{align}
    Z = \sqrt{\frac{\mu_r \mu_0}{\epsilon_r \epsilon_0}} \frac{\ln(R_a / R_i)}{2 \pi} = \sqrt{\frac{L'}{C'}} &= Z_{frei} \frac{\ln(R_a / R_i)}{2 \pi}\\
    &= Z_{frei}^{vak} \sqrt{\frac{\mu_r}{\epsilon_r}}\frac{\ln(R_a / R_i)}{2 \pi} \label{wellenwiderstand}
\end{align}
\sgl{wellenwiderstand}{1.15}

Erhöht man die Verzögerungszeit durch ein Material mit größerem $\epsilon_r$, so wird $Z$ kleiner, Erhöht man die Induktivität (sei es durch Ferritkern, Windungen oder Materialwahl), so wird der Wellenwiderstand $Z$ größer.

\section{C}
In diesem Fall ($R_A = Z$), ist $R_{in} = Z = R_A$ unabhängig von der Länge des Kabels.

\section{D}
Gegeben sind die Werte:
\begin{align}
    &R_A / R_I = 2.3\\
    &\epsilon_r = 1.5\\
    &\mu_r = 1.5\\
    &Z_{frei}^{vak} \approx 377\Omega\label{zfrei}
\end{align}
\sgl{zfrei}{1.16}
Im Verlustfreien Idealfall ist der Wellenwiderstand durch (\ref{wellenwiderstand}) gegeben. Hier ist dann $Z \approx 50\Omega$.
Die Phasengeschwindigkeit ist gegeben durch (\ref{vph}), wobei hier $c_0 = 3 \cdot 10^8 m s^{-1}$ genommen wird, und wird dann zu $v_{ph} = 2 \cdot 10^8 m s^{-1}$.

Die Verzögerung pro Meter ist dann einfach die reziproke Geschwindigkeit, also $\tau = v_{ph}^{-1} = \frac12 10^{-8} s m^{-1}$.

\chapter{Versuchsdurchführung}

\chapter{Auswertung}

\chapter{Fazit}

\end{document}
